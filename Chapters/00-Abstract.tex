\thispagestyle{plain}
\chapter*{Resumen}

% Versión en español del resumen
\guideinfo{En la sección \textit{Resumen}, presenta un resumen conciso de tu proyecto, destacando los puntos clave. Comienza con una breve declaración del problema u objetivo, seguida de una descripción de tu enfoque o metodología. Resume los principales resultados o conclusiones, resaltando su importancia o implicaciones. Concluye con una o dos frases sobre la contribución global o el impacto de tu trabajo. El resumen debe ser claro y conciso, idealmente entre 150 y 250 palabras, para que los lectores comprendan rápidamente tu investigación y su relevancia.}

\keywordses{Palabra clave A, Palabra clave B, Palabra clave C.}

\MediaOptionLogicBlank

\pdfbookmark[1]{Abstract}{abstract}
\chapter*{Abstract}
\guideinfo{In the \textit{Abstract} section, provide a concise summary of your project, highlighting the key points. Begin with a brief statement of the problem or objective, followed by a description of your approach or methodology. Summarise the main results or findings, emphasising their significance or implications. Conclude with a sentence or two on the overall contribution or impact of your work. Keep the abstract clear and focused, ideally within 150-250 words, to give readers a quick understanding of your research and its importance.}

\keywordsen{Keyword A, Keyword B, Keyword C.}

\MediaOptionLogicBlank